\RequirePackage{plautopatch}  % pLaTeX または upLaTeX のとき
%\documentclass[uplatex,dvipdfmx,titlepage,a4j]{jsarticle}% upLaTeX のとき
\documentclass[dvipdfmx,titlepage,a4j]{jsarticle}  % pLaTeX のとき
\usepackage{listings,jvlisting}
\usepackage{amsmath,amssymb}
\usepackage{graphicx}
\usepackage[yen]{okuverb}
\usepackage{r04ec-exp}
\usepackage{here}
\usepackage{ascmac}
\usepackage{fancybox}
\usepackage{fancyvrb}
\usepackage{fancyhdr}
\usepackage{lastpage}

\fancypagestyle{foot}
{
\fancyhead[C]{トランジスタの増幅回路とR-L-C共振回路}
\fancyfoot[C]{\thepage / \pageref{LastPage}}
\renewcommand\headrulewidth{0.4pt}
}

%ここからソースコードの表示に関する設定
\lstset{
  language={C++},
  basicstyle={\ttfamily},
  identifierstyle={\small},
  commentstyle={\smallitshape},
  keywordstyle={\small\bfseries},
  ndkeywordstyle={\small},
  stringstyle={\small\ttfamily},
  frame={tb},
  tabsize={2},
  breaklines=true,
  columns=[l]{fullflexible},
  numbers=left,
  xrightmargin=0zw,
  xleftmargin=3zw,
  numberstyle={\scriptsize},
  stepnumber=1,
  numbersep=1zw,
  lineskip=-0.5ex
}

\renewcommand{\lstlistingname}{リスト}
%ここまでソースコードの表示に関する設定

\title{トランジスタの増幅回路とR-L-C共振回路}
% 学年・番号
\grade{4年42番}%
% 氏名
\author{鷲尾 優作}
% 班(後期は班に分かれて実験をする.そのときは,ここに班番号を記入する.)
\team{A班}
% 提出日
\date{2022年6月16日}
% 実験日
\expdate{2022年5月26日,6月2日,6月9日}
% 共同実験者
% グループに分かれて実験をするテーマでは,グループメンバーの番号名前を書く.
\coauthor{}
%
%記載例:
%\coauthor{%
%  2番 & 新潟 花子\\
%  11番 & 三条 次郎}
%%

\begin{document}
\pagestyle{foot}

\maketitle

\section{背景・目的}
電子制御工学科4年前期(4年42番)のトランジスタの増幅回路とR-L-C共振回路実験について報告する.

半導体を用いて製造される能動素子であるトランジスタは, 「増幅作用」をもつが, 信号を増幅させる際,周辺回路の構成によって増幅特性に変化を生じる.
一方受動素子であるインダクタとコンデンサを用いる, LC回路は特定の周波数の信号を生成したり,複雑な信号から特定の周波数の信号だけを抽出するのに使用できる.

このレポートでは等価回路による理論値の算出,
バイポーラトランジスタを用いた増幅回路の実験, R-L-C回路共振回路の動作実験をそれぞれ行い,
理論値と実測値を比較しそれぞれの特性を確認した.

\section{トランジスタの増幅回路とその特性}
今回の実験では,エミッタ増幅回路をトランジスタの周辺回路として用いる.
トランジスタに直流バイアス成分を加え活性領域に動作点を設定し,
コンデンサC1を介して交流信号を重ねてトランジスタのベース端子に入力
増幅された交流信号をコレクタ端子に接続されたコンデンサC2を介して取り出す仕組みである.

図\ref{fig:tr}にエミッタ増幅回路を示す.

回路図上の各定数は以下のように設定した.

$R_1 = 22$ k$\Omega$, $R_2 = 100$ k$\Omega$, $R_3 = 2$ k$\Omega$, $R_4 = 10$ k$\Omega$,
$R_L = 10$ k$\Omega$, $C_1 = 0.1$ $\mu$F, $C_2 = 10$ $\mu$F, $E = 15$ V$R_1 = 22$ k$\Omega$, $R_2 = 100$ k$\Omega$, $R_3 = 2$ k$\Omega$, $R_4 = 10$ k$\Omega$,
$R_L = 10$ k$\Omega$, $C_1 = 0.1$ $\mu$F, $C_2 = 10$ $\mu$F, $E = 15$ V

\subsection{増幅回路の理論特性}
エミッタ増幅回路がどのような特性を示すか推定するため,電気的等価回路を作成し,
代表的な理論値を等価回路から導かれる計算式を用いて計算した.

\subsubsection{バイアス値}
トランジスタにおけるバイアス値は,交流信号を出力する際に基準となる電位の値であり,動作点を決定する重要な
パラメータである.バイアス値を導出する際には回路内の直流分のみを抽出した,「バイアス回路」を考える.

以下図 に実験で使用した回路をバイアス回路化したものを示す.

ここで,動作点を決定するために必要な各電圧$V_{BE}$, $V_{E}$, $V_{C}$を求める計算式は以下のとおりである.

\subsubsection{電圧利得}
電圧利得は,入力信号に対して出力信号の比が何dB上昇したかを表す.増幅回路そのものの性能を示すうえで重要な
パラメータである.電圧利得を算出するためには,まず結合コンデンサのインピーダンスを0とした交流分の等価回路を考える.

以下図 に実験で使用した回路の結合コンデンサのインピーダンスを0とした交流分の等価回路を示す.

\subsubsection{低域カットオフ周波数}
低域カットオフ周波数は,入力信号の周波数が減少した際に増幅が正常に行われなくなる現象において,その下限周波数の目安となる
パラメータである.低域カットオフは結合コンデンサ$C_1$によって生じるため,算出するためには$C_2$
を無視した簡易等価回路を考える.

以下図 に$C_2$を無視した簡易等価回路を示す.

\subsubsection{入力インピーダンス}
回路全体の入力インピーダンスを算出する.


\subsubsection{出力インピーダンス}
回路全体の出力インピーダンスを算出する.

\subsubsection{出力波形の歪み}

\subsection{増幅回路の作成と実測}

\subsubsection{バイアス電圧の測定}

\subsubsection{電圧利得の測定}

\subsubsection{増幅率の周波数特性}

\subsubsection{入力インピーダンスの測定}

\subsubsection{出力インピーダンスの測定}

\subsubsection{増幅率の測定と波形歪みの観測}

\section{R-L-C共振回路とその特性}

\subsection{定数と理論特性}

\subsubsection{アドミッタンスと共振周波数}

\subsubsection{アドミッタンスループ}

\subsubsection{品質係数Q値の算出}

\subsubsection{R, L, C値の算出方法}

\subsection{R-L-C回路のアドミッタンス特性測定と定数の算出}

\subsubsection{アドミッタンス特性の測定}

\subsubsection{定数の算出}

\subsubsection{グラフの追記}

\section{課題}

\subsubsection{エミッタ接地,ベース接地,コレクタ接地の各増幅回路の特徴についてまとめよ.}

\subsubsection{共振回路の応用用途にはどのようなものがあるか調査し,それぞれについてまとめよ.}

\section{感想}

\begin{thebibliography}{99}
  \bibitem{umeda} 梅田 幹雄、実験テキスト「トランジスタの増幅回路とR-L-C共振回路」、(2022年)
\end{thebibliography}

\end{document}